%%%%%%%%%%%%%%%%%%%%%%%
%% Set theory macros %%
%%%%%%%%%%%%%%%%%%%%%%%
	
% Allows something to be called outside math mode with the right spacing
\newcommand{\setdefinition}[1]{\ensuremath{#1}\xspace}

% Important sets. Can be called also in text mode.
	\newcommand{\naturals}{\setdefinition{\mathbb{N}}} % Set of natural numbers
	\newcommand{\integers}{\setdefinition{\mathbb{Z}}} % Set of integer numbers
	\newcommand{\rationals}{\setdefinition{\mathbb{Q}}} % Set of rational numbers
	\newcommand{\reals}{\setdefinition{\mathbb{R}}} % Set of real numbers
	\newcommand{\complexs}{\setdefinition{\mathbb{C}}} % Set of complex numbers
	\newcommand{\integersMod}[1]{\setdefinition{\mathbb{Z}_{#1}}} % Set/group/ring of integers mod #1


% Set-theoretic operations

	\newcommand{\Powerset}[1]{\mathcal{P}(#1)} % Powerset of #1
	\newcommand{\modclass}[2]{#1 \; (\text{mod } #2)} % Equivalence class of integers mod #2 corresponding to representative #1
	\newcommand{\suchthat}[2]{\left\{#1 \: \colon \: #2\right\}} % Set of elements #1 such that condition #2 holds 
	
	\newcommand{\domain}[1]{\operatorname{dom}#1} % Domain of function/morphism #1
	\newcommand{\codomain}[1]{\operatorname{codom}#1} % Codomain of function/morphism #1
	\newcommand{\support}[1]{\operatorname{supp}#1} % Support of function/morphism #1
	\newcommand{\cosupport}[1]{\operatorname{cosupp}#1} % Cosupport of function/morphism #1
	
	\newcommand{\restrict}[2]{\left. #1 \right\vert_{#2}} % Restriction of function/morphism #1 to subset/subobject #2

	\newcommand{\inject}{\hookrightarrow} % Set-theoretic injection
	\newcommand{\Irreps}[1]{\operatorname{Irr}[#1]} % Set of irreps for group #1

	\newcommand{\closure}[1]{\operatorname{Cl}({#1})} %Standard notation for closure operators			


% Non-standard Analysis
	\newcommand{\starNaturals}{\setdefinition{\nonstd{\naturals}}} % Set of non-standard naturals
	\newcommand{\starIntegers}{\setdefinition{\nonstd{\integers}}} % Set of non-standard integers
	\newcommand{\starRationals}{\setdefinition{\nonstd{\rationals}}} % Set of non-standard rationals
	\newcommand{\starComplexs}{\setdefinition{\nonstd{\complexs}}} % Set of non-standard complex numbers
	\newcommand{\starReals}{\setdefinition{\nonstd{\reals}}} % Set of non-standard realsnumbers
	
	\newcommand{\Infinitesimals}{\mathbb{I}} % Ring of infinitesimal hyperreals
	\newcommand{\Limited}{\mathbb{L}} % Ring of limited hyperreals

	\newcommand{\nonstd}[1]{{^\star\!#1}} % Non standard extension

	\newcommand{\near}{\sim} % Limited distance relation symbol
	\newcommand{\infnear}{\simeq} % Infinitesimal distance relation symbol
	\newcommand{\Halo}[1]{\operatorname{Hal}(#1)} % Halo (monad) of a hyperreal.
	\newcommand{\Galaxy}[1]{\operatorname{Gal}(#1)} % Galaxy of a hyperreal
	\newcommand{\Shadow}[1]{\operatorname{Sh}(#1)} % Shadow of a hyperreal
%% Legacy: For macro-maniacs freaks
%	\newcommand{\set}[1]{\{#1\}} 
%	% Standard notation for sets.
%	\newcommand{\cat}[1]{\mathcal{#1}}
%	% Standard notation for categories
%	\newcommand{\couple}[2]{(#1,#2)} 
%	%Classical notation for set-theoretic couples
%	\newcommand{\relation}[1]{#1} 
%	% Relation between sets
%	\newcommand{\congruenceclass}[2]{[#2]_{#1}}
%	%Standard notation for congruence/equivalence classes
%	\newcommand{\quotient}[2]{#1/#2} 
%	%Quotient of algebraic structures		
% 	\newcommand{\substruct}[2]{{\operatorname{Sub}_{#1}(#2)}} % Set of subalgebras of a given algebra
